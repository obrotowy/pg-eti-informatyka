\documentclass{article}
\usepackage[utf8]{inputenc}
\usepackage{polski}
\usepackage[margin=2cm]{geometry}
\usepackage{graphicx}
\usepackage{float}
\usepackage{booktabs}
\usepackage{array}


\setcounter{tocdepth}{2}
\let\origfigure\figure
\let\endorigfigure\endfigure
\renewenvironment{figure}[1][2] {
    \expandafter\origfigure\expandafter[H]
} {
    \endorigfigure
}


\title{Sieci Komputerowe}
\author{Emilian Zawrotny\\ Na podstawie materiałów Prof. Krzysztofa Nowickiego, \\
    dr. Krzysztofa Gierłowskiego}
\date{\today}

\begin{document}
\maketitle
\pagebreak
\tableofcontents
\pagebreak

\section{Systemy Teleinformatyczne}
    \subsection{Porównanie sieci WAN i LAN}
        \begin{center}
        \begin{tabular}{c@{\hspace{2cm}}c}
        \begin{minipage}[t]{0.4\textwidth}
            \begin{center}
            Sieci lokalne \textbf{LAN}
            \end{center}
            \begin{itemize}
                \item Niewielkie opóźnienia (mikrosekundy)
                \item Znikoma stopa błędu
                \item Większa przepustowość u użytkownika końcowego pomimo znacznie mniejszej ogólnej przepustowości sieci
                \item Jeden zarządzający
                \item Łatwe i tanie w implementacji
            \end{itemize}
        \end{minipage}
    &
        \begin{minipage}[t]{0.4\textwidth}
            \begin{center}
            Sieci rozległe \textbf{WAN}
            \end{center}
            \begin{itemize}
                \item Znaczne opóźnienia (rzędu dziesiątek milisekund)
                \item Znacząca stopa błędu
                \item Większa zawodność, większa podatność na uszkodzenia łączy
                \item Wielu zarządzających
                \item Większa przepustowość całej sieci, jednak podzielona na większą ilość użytkowników
                \item Duże koszty implementacji
            \end{itemize}
        \end{minipage}
        \end{tabular}
        \end{center}

        \textbf{Problem:} Używamy tej samej architektury TCP/IP w obu rodzajach sieci - a co za tym idzie używamy mechanizmów WANowskich w sieciach lokalnych. Nasze routery na darmo liczą sumy kontrolne każdego pakietu, mimo iż ryzyko błędu jest niemalże zerowe, jest to marnotractwo mocy i wprowadzanie zbędnych opóźnień. \linebreak

        Internet jest szczególną implementacji sieci WAN - opartą o architekturę TCP/IP, ze wstępnie zdefiniowanym zestawem protokołów i usług działających w architekturze klient-serwer.

    \subsection{Klasyfikacja sieci ze względu na ich strukturę}
        \begin{center}
        \begin{tabular}{c@{\hspace{2cm}}c}
        \begin{minipage}[t]{0.4\textwidth}
            \begin{center}
            Struktura hierarchiczna
            \end{center}
            \begin{itemize}
                \item Zarządca widzący całą sieć, może ustalać globalnie routing na podstawie wiedzy o całej sieci - znaczący wpływ na wydajność
                \item Single Point of Failure
                \item Niemożliwe do zrealizowania w globalnej sieci przez względy polityczne
                \item Słabo się skaluje z ilością węzłów
            \end{itemize}
        \end{minipage}
    &
        \begin{minipage}[t]{0.4\textwidth}
            \begin{center}
            Struktura rozproszona (\textbf{Używana w internecie})
            \end{center}
            \begin{itemize}
                \item Każdy węzeł sam podejmuje decyzje o tym, jak odbywa się routing. Robi to przy mocno ograniczonej wiedzy nt. całej sieci
                \item Łatwa skalowalność na dużą liczbę węzłów
                \item Brak Single Point of Failure
            \end{itemize}
        \end{minipage}
        \end{tabular}
        \end{center}
        
    \subsection{Komutacja - sposób zestawiania fizycznych połączeń}
        \subsubsection{Komutacja kanałów}
        \begin{itemize}
            \item Zestawianie połączenia przed rozpoczęciem transmisji danych - dodatkowe opóźnienie
            \item Łącze fizyczne użytkowane na wyłączność
            \item Wyłącznie opóźnienia propagacji - przydatne przy stałym natężeniu strumienia np. w transmisjach multimedialnych
            \item Niezmieniona kolejność pakietów
        \end{itemize}
        \begin{figure}
        \begin{center}
        \includegraphics[width=0.4\textwidth]{assets/channel-comutation.png}
        \end{center}
        \end{figure}
        \subsubsection{Komutacja pakietów w połączeniu wirtualnym}
        Można wulgarnie porównać do protokołu TCP. Działa analogicznie jak komutacja kanałów, z tą różnicą że dane są podzielone na pakiety.
        \subsubsection{Komutacja pakietów w sieci datagramowej / szybka komutacja pakietów}
        Można wulgarnie porównać do protokołu UDP. W ten sposób de facto działa protokół IP i ew. protokół TCP w warstwie wyższej zajmuje się sprawdzaniem kolejności pakietów.
        \begin{itemize}
            \item Bezpołączeniowo - węzeł ufa, że inny węzeł jest dalej dostępny
            \item Każdy pakiet jest trasowany indywidualnie
            \item Kolejność otrzymywania pakietów jest niedeterministyczna
            \item Szybka komutacja nie stosuje nawet sum kontrolnych - ryzyko marnowania łącza do dalszego przesyłu uszkodzonej ramki.
        \end{itemize}
        \begin{figure}
            \begin{center}
                \includegraphics[width=0.4\textwidth]{assets/datagram-comutation.png}
            \end{center}
        \end{figure}
        \subsubsection{Współdzielenie medium przy komutacji kanałów}
            \begin{itemize}
                \item \textbf{FDM} (Frequency Division Multiplexing) - podział na pasma częstotliwości np. telewizja, radio, Wi-Fi (kilka sieci w jednym pomieszczeniu)
                \item \textbf{TDM} (Time Division Multiplexing) - podział w czasie
            \end{itemize}
            \begin{figure}
                \begin{center}
                    \includegraphics[width=0.4\textwidth]{assets/TDM-FDM.png}
                \end{center}
                \caption{Wizualizacja obu podziałów}
        \end{figure}
        \subsubsection{Problem}
            Obecnie większość ruchu internetowego to transmisje multimedialne - do których idealnie sprawdziłaby się komutacja kanałów, lecz internet wykorzystuje komutacje pakietów, co skutecznie obniża efektywność wykorzystania łącza.
\pagebreak
\section{Warstwowa architektura sieci}
    \subsection{Jaki jest problem i po co jest?}
        \begin{itemize}
            \item Potrzeba kompatybilności różnorakiego sprzętu - np. Wi-Fi i Ethernet
            \item Niezależny rozwój różnych obszarów - osobno rozwijane kwestie fizyczne, osobno aplikacje
            \item Ogranicza złożoność sieci
        \end{itemize}
    \subsection{Enkapsulacja danych}
        Dane są opakowywane w nagłówki kolejnych warstw stosu sieciowego.
        \begin{figure}
            \begin{center}
                \includegraphics[width=0.4\textwidth]{assets/encapsulation.png}
            \end{center}
            \caption{Przykład enkapsulacji danych na przykładzie TCP/IP}
        \subsubsection{Problem}
            Nagłówki w sieci TCP/IP generują narzut łącznie 58 bajtów - to sprawia, że wysyłając jeden bajt, by otrzymać prędkość na poziomie 64kb/s potrzebujemy przepustowości łącza na poziomie 3.8 Mb/s. Efektywność łącza u automatyków jest wtedy znikoma - nawet 98\% wysyłanych bitów mogą stanowić nagłówki, a nie same wysyłane dane. Ktoś to musi zmienić.
        \end{figure}
    \subsection{Zasady współpracy warstw między sobą}
        \begin{itemize}
            \item Każda warstwa pełni funkcję warstwy transportowej dla warstwy powyższej.
            \item O protokołach wykorzystywanych w danej warstwie mówi się potocznie \textit{Protokoły warstwy N-tej}
        \end{itemize}
    \subsection{Porównanie modelu ISO/OSI i TCP/IP}
        \begin{figure}
            \begin{center}
                \includegraphics[width=0.4\textwidth]{assets/tcp-osi.png}
            \end{center}
        \end{figure}
        \subsubsection{Główne różnice}
        \begin{itemize}
            \item \textbf{ISO/OSI} jest modelem abstrakcyjnym, służącym do projektowania faktycznych sieci. \textbf{TCP/IP} to z kolei konkretne rozwiązanie sieciowe,
            \item Model \textbf{ISO/OSI} definiuje tylko tryb połączeniowy w warstwie transportowej, \textbf{TCP/IP} oferuje tryb połączeniowy (\textbf{TCP}) i bezpołączeniowy (\textbf{UDP}),
            \item Liczba warstw,
            \item Model \textbf{TCP/IP} definiuje wyłącznie tryb bezpołączeniowy w warstwie sieciowej, \textbf{ISO/OSI} nie.
            \item Sposób ich realizacji,
            \item Zasady nawiązywania połączeń między stacjami.
        \end{itemize}
        \subsubsection{Podobieństwa}
        \begin{itemize}
            \item Idea stosu protokołów
            \item Górne warstwy opisują aplikacje
            \item Dolne warstwy opisują komunikację
        \end{itemize}
    \subsection{Szczegółowe omówienie modelu ISO/OSI}
    Pierwsze trzy warstwy odpowiadają za komunikację między stacjami, biorą więc udział również między węzłami pośrednimi. Warstwy 4-7 to warstwy aplikacyjne - te występują jedynie na stacjach końcowych.
    \subsubsection{Warstwa fizyczna - np. skrętka, światłowód, eter, magistrala CAN}
    \begin{itemize}
        \item Odpowiada za przesył bitów między stacjami
        \item Sygnały, kodowania, modulacje - to wszystko dzieje się w tej warstwie
    \end{itemize}
    \subsubsection{Warstwa łącza danych - np. Ethernet, FDDI, IEEE 802.11 (Wi-Fi)}
    \begin{itemize}
        \item Tworzenie ramek, zapewnianie ich przepływu
        \item Zapewnienie niezawodności w zawodnynym medium transmisyjnym
    \end{itemize}
    \subsubsection{Warstwa sieciowa - np. IP, ICMP}
    \begin{itemize}
        \item Routing
        \item Łaczenie sieci między sobą
        \item Przeźroczysty przekaz danych - bez ingerencji w nie
        \item Segmentacja i desegmentacja pakietów
    \end{itemize}
    \subsubsection{Warstwa transportowa - np. TCP, UDP}
    \begin{itemize}
        \item Zapewnenie niezawodnego przesyłu między węzłami końcowymi
        \item Zapewnienie retransmisji utraconych pakietów
    \end{itemize}
    \subsubsection{Warstwa sesji - Gniazda sieciowe, SOCKS, RPC}
    \begin{itemize}
        \item Nawiązywanie połączeń
        \item Dialog między procesam
    \end{itemize}
    \subsubsection{Warstwa prezentacji - np. MIME, TLS, SSL}
    \begin{itemize}
        \item Przekształcenie danych "ludzkich" na komputerowe.
        \item Zapewnienie atrybutów bezpieczeństwa
        \item Kompresja danych
    \end{itemize}
    \subsubsection{Warstwa aplikacji - np. HTTP, SMTP, SSH}
    Warstwa świadczenia konkretnej usługi np. rozsyłanie poczty czy transmisja plików
    \subsection{Architektura TCP/IP}
    \subsubsection{Warstwa dostępu do sieci}
    \begin{itemize}
        \item Odpowiednik dwóch pierwszych warstw modelu ISO/OSI
        \item W przypadku sieci LAN: Ethernet lub Wi-Fi
        \item W przypadku sieci rozległych: PPP, SLIP, X.25, Frame Relay
    \end{itemize}
    \subsubsection{Warstwa międzysieciowa}
    \begin{itemize}
        \item W tej warstwie funkcjonuje protokół \textbf{IP}
        \item Przenosi dane między warstwą transportową a dostępu do sieci w strukturach zwanych \textbf{datagramami}.
    \end{itemize}
    \subsubsection{Warstwa transportowa}
    Protokoły TCP i UDP
    \subsubsection{Warstwa aplikacji}
    Protokoły HTTP, SMTP, SSH i wiele, wiele innych.
\pagebreak
\section{Charakterystyki metod dostępu do medium}
\subsection{Metody zwielokrotniania dostępu do medium komunikacyjnego}
\begin{itemize}
    \item \textbf{FDM} (Frequency Division Multiplexing) - podział na pasma częstotliwościowe. Wykorzystywany w Wi-Fi w obrębie wielu sieci (tj. każda sieć ma swoje pasmo)
    \item \textbf{TDM} (Time Division Multiplexing) - podział w czasie tj. każdy ma swój kwant czasu na nadawanie. Wykorzystywany w Wi-Fi w obrębie jednej sieci (tj. każdy węzeł w sieci ma swój czas na nadawanie)
        \begin{itemize}
            \item \textbf{STDM} - synchroniczny, bezkolizyjny, problemy z synchronizacją
            \item \textbf{ATDM} - asynchroniczny, kontrolowany lub rywalizacyjny
        \end{itemize}
    \item \textbf{CDM} (Code Division Multiplexing) - zwielokrotniania kodowe, wydobywanie sygnału spośród dużego szumu np. w łączności satelitarnej
    \item \textbf{WDM} (WaveLength Division Multiplexing) - podział na długość fali wykorzystywany w światłowodach 
\end{itemize}
\subsection{Algorytmy dostępu}
\begin{tabular}{>{\raggedright\arraybackslash}p{4cm} | >{\raggedright\arraybackslash}p{4.5cm} | >{\raggedright\arraybackslash}p{4.5cm} | >{\raggedright\arraybackslash}p{4.5cm}}
    \toprule
    & \textbf{CSMA/CD} - Ethernet & \textbf{Tokenowe} - Token Ring/Token Bus & \textbf{CSMA/CA} - Wi-Fi \\
    \midrule
    Rozmiar ramki & 64B - 1500B & do 18kB & do 2312 B \\
    \midrule
    Kolizje & Klasyczny CSMA/CD tak, full-duplex nie & Nie & Tak (można odpytywać PCF, wówczas mamy okres czasu bezkolizyjny) \\
    \midrule
    Rodzaj algorytmu & rywalizacyjny & tokenowy & rywalizacyjny \\
    \midrule
    Opóźnienie w dostępie do medium & niedeterministyczne (zależne od obciążenia) & stałe & niedeterministyczne (zależne od obciążenia) \\
    \midrule
    Priorytety & Pierowtnie nie, wprowadzone w 802.1Q & Tak & Tak (zdefiniowane w 802.11e) \\
    \midrule
    Obsługa ruchu synchronicznego & Nie & Tak & Tak (okresowe wysyłanie Beaconów) \\
    \midrule
    Mechanizmy do różnicowania jakości obsługi aplikacji & klasyczna wersja nie, 802.1Q dokłada elementy QoS (priorytety) & Tak & Tak (802.11e) \\
    \midrule
    Inne & VLANy, sprawiedliwa obsługa stacji & rezerwacje tokena, Token bus ma skomplikowany przepływ tokena & DAC - distributed access control \\
    \bottomrule
\end{tabular}
\subsection{Problem}
    Zmienne opóźnienie znacząco utrudnia transmisję danych multimedialnych. Ktoś musi wymyśleć nowy algorytm - albo wskrzezić Token Busa by ten wyparł Ethernet.
\pagebreak
\section{Ethernet 10/100/1000Mbps}
\textbf{Ethernet} to technologia warstwy dwóch pierwszych warstw modelu \textbf{ISO/OSI}. W tej sekcji omówimy standardy:
\begin{itemize}
    \item 10Mbps Ethernet 
    \item 100Mbps Fast Ethernet
    \item 1000Mbps Gigabit Ethernet
\end{itemize}
\subsection{Sposoby kodowania}
\subsubsection{10Mbps Ethernet - kodowanie sygnału Manchester}
\begin{itemize}
    \item W środku bitu (zbocze opadające zegara) \textbf{ZAWSZE} następuje zmiana stanu,
    \item Na początku bitu (zbocze narastające zegara) może nastąpić zmiania bitu, zależnie od nadawanych danych.
    \item Zmiana napięcia przy zboczu opadającym (w środku bitu) symbolizuje bit:
        \begin{itemize}
            \item high $\rightarrow$ low: 0
            \item low $\rightarrow$ high: 1
        \end{itemize}
        (W konwencji Ethernetowej IEEE 802.3, pierwotna konwencja Manchester zakładała odwrotnie)
    \item Wymuszona zmiana stanu co cykl umożliwia łatwą synchronizację stacji, a także usuwa problem stałej składowej.
\end{itemize}
\begin{figure}
    \begin{center}
        \includegraphics[width=0.4\textwidth]{assets/manchester.png}
    \end{center}
    \caption{Kodowanie Manchester by Stefan Schmidt}
\end{figure}
\textbf{Problem: }Wymaga częstotliwości pasma będącej dwukrotnością zadanej przepustowości łącza. Pasmo 200MHz w skrętce dla 100Mbps Ethernetu było wówczas niemożliwe do realizacji.
\subsubsection{Kodowanie danych $x$b/$y$b}
Grupa $x$ bitów jest zamieniana na grupę $y > x$ bitów wymuszając odpowiednio częstą zmianę stanu.

Po co?
\begin{itemize}
    \item Zapewnienie synchronizacji stacji
    \item Wykrywanie awarii łącza (np. dla kodowania \textbf{4b/5b} wystąpienie 4 zer pod rząd oznacza awarię)
    \item W technice światłowodowej zapobiega nadmiernemu przegrzewaniu (laser nie świeci nieustannie nawet podczas przesyłania samych jedynek)
\end{itemize}
\subsubsection{Fast Ethernet - kodowanie sygnału \textbf{MLT-3}}
\begin{itemize}
    \item Wykorzystuje kodowanie danych \textbf{4b/5b}
    \item Trójpoziomowe
    \item Zmiana poziomu na zboczu narastającym zegara oznacza bit=1
\end{itemize}
\begin{figure}
    \begin{center}
        \includegraphics[width=0.4\textwidth]{assets/mlt3.png}
    \end{center}
\end{figure}
Potrzebna częstotlwiość pasma jest 4-krotnie mniejsza od zadanej przepustowości. Trzeba jednak brać pod uwagę narzut wynikający z kodowania danych \textbf{4b/5b} - by przesłać 100Mb danych, musimy rzeczywiście wysłać ich 125Mb.
$$f = \frac{100Mb/s \cdot \frac{5}{4}}{4} = 31.25MHz$$
\subsubsection{Gigabit Ethernet po skrętce (1000BASE-T) - kodowanie 4D-PAM5}
\begin{itemize}
    \item Wykorzystujemy 4 niezależne kanały po 5 poziomów każdy, działające w trybie Full-Duplex
    \item Kodowanie danych \textbf{8B1Q4} wykorzystywane w tym standardzie nie było omawiane na wykładzie, wystarczy wiedzieć, że nie generuje dodatkowego narzutu 
    \item Potrzebujemy 8-krotnie mniejszą częstotliwość względem przepustowości
    $$f = \frac{1000Mb/s}{8} = 125MHz$$
\end{itemize}
\subsection{Ramka Ethernetowa}
\begin{figure}
    \begin{center}
        \includegraphics[width=0.6\textwidth]{assets/ethernet-frame.png}
    \end{center}
    \caption{Ramka Ethernetowa}
    \textbf{Uwaga:} W niektórych źródłach można znaleźć informację, że minimalny rozmiar danych to 42 bajty - wynosi on tyle, gdy korzystamy ze standardu 802.1Q który dorzuca nam dodatkowe, 4 bajtowe pole.
\end{figure}
Preambuła i pole startu służą do synchronizacji między stacjami, faktyczną ramkę w znaczeniu warstwy drugiej stanowią następne pola.
\subsubsection{Skąd minimalna długość danych?}
Podczas projektowania standardu Ethernet założono zasięg sieci $2500$ m. Ówczesne łącza koncentryczne wymuszały długość segmentu $500$ m. A algorytm \textbf{CSMA/CD} umożliwia wykrywanie kolizji wyłącznie wtedy, gdy węzeł sam jeszcze nadaje.
\begin{center}
\begin{tabular}{c@{\hspace{2cm}}c}
\begin{minipage}[t]{0.35\textwidth}
    $s = 500m \\
    v = \frac{2}{3}c$ - założona prędkość rozchodzenia się sygnału w koncentryku
    $$t_{\mbox{segment}} = \frac{s}{v} = \frac{500 m}{2\cdot10^8} = 2.5\mu s$$
    $$t_{\mbox{w obie strony}} = 2 \cdot (5\cdot2.5\mu s + 4\cdot 2\mu s + 2\cdot 1.5 \mu s) = 51 \mu s$$
\end{minipage}
&
\begin{minipage}[t]{0.5\textwidth}
    \begin{figure}
        \includegraphics[width=\textwidth]{assets/segments.png}
    \end{figure}
\end{minipage}
\end{tabular}
\end{center}
Z tego wynika, że dwukrotność czasu propagacji musi wynosić $51 \mu s$. Zaokrąglono więc tą wartość do $51.2 \mu s$ i skonstruowano równanie:
$$51.2 \cdot 10^{-6}s \cdot 10^{7} \frac{b}{s} = 512b = 64B$$
i uznano je za świętość. Jako że nagłówki zajmują 18 bajtów, żeby cała ramka była niekrótsza niż 64 bajty, rozmiar danych trzeba ograniczyć z dołu do 46 bajtów.

\textbf{Problem: }Wraz ze wzrostem przepustowości maleje zasięg sieci, choć zasięg w Fast Ethernet był jescze akceptowalny, tak dla Gigabitowego Ethernetu była potrzeba porzucenia \textbf{CSMA/CD} na rzecz \textbf{Full-Duplex} - mimo to świętość pozostała, na rzecz kompatybilności wstecznej.
\subsubsection{Skąd maksymalna długość danych?}
Założenia:
\begin{itemize}
    \item Możliwość wysłania strony tekstu w jednej ramce
    \item Możliwość przechowania całej ramki w RAMie (w latach 70' to wcale nie było takie oczywiste)
    \item Ograniczenie liczby retransmisji
    \item BER ówczesnych koncentryków na poziomie $10^{-8}$
\end{itemize}
Przy tych założeniach uznano wartość 1500 bajtów za złoty środek pomiędzy ilością wysyłanych danych, a ilością błędów w transmisji.

Obecnie skrętki oferują BER na poziomie $10^{-11}$, światłowody jeszcze niższy, więc ramki możnaby było wydłużyć, lecz usuwa to kompatybilność wsteczną.

Istnieją też ramki "Jumbo", lecz te nie są ustandaryzowane i bywają problematyczne, gdy korzystamy ze sprzętów sieciowych różnych firm w obrębie jednej sieci.
\pagebreak
\section{Ethernet 10/40/100/400Gbps}
\subsection{Ethernet 10Gbps}
Założenia:
\begin{itemize}
    \item Osobne warstwy fizyczne dla sieci LAN i WAN
    \item Pierwotnie tylko światłowód, później jednak powstał standard \textbf{10GBASE-T}
\end{itemize}
\subsubsection{Implementacje 10Gb/s Ethernetu}
Światłowodowe: \textbf{10GBASE-\{S,L,E\}\{R, W\}}, \textbf{10GBASE-LX4}\\
Miedziane: \textbf{10GBASE-T} (skrętka), \textbf{10GBASE-CX4} (koncentryk) \\
Backplane (styki na PCB): \textbf{10GBASE-K\{X4, R\}}
\begin{itemize}
    \item \textbf{S, L, E} - długość fali 
        \begin{itemize}
            \item Short: 850 nm
            \item Long: 1310 nm
            \item Extra long: 1550 nm
        \end{itemize}
    \item \textbf{R} - LAN PHY, \textbf{W} - WAN PHY
    \item \textbf{X} - transmisja równoległa (korzysta z kodowania \textbf{8b/10b})
    \item 4 - 4 pary kabla
    \item \textbf{BASE} - Baseband, TDM
\end{itemize}
\subsection{IEEE 802.3ba - 40Gbps i 100Gbps}
\begin{itemize}
    \item Pierowtnie 4 światłowody po 10Gb/s każdy (QSFP) lub 10 (CXP)
    \item Później możliwość osiągnięcia 100Gb/s przy użyciu 4 światłowodów po 25Gbps każdy
\end{itemize}
\subsubsection{Implementacje 100 Gb/s Ethernet}
\begin{center}
\begin{tabular}{c@{\hspace{2cm}}c}
\begin{minipage}[t]{0.35\textwidth}
    \begin{center}
        \textbf{100GBASE-SR10}
        \begin{itemize}
            \item 10 $\times$ 10Gb/s
            \item Światłowód wielomodowy
            \item Max. odległość: 100m
        \end{itemize}
    \end{center}
\end{minipage}
&
\begin{minipage}[t]{0.5\textwidth}
    \begin{center}
        \textbf{100GBASE-\{L,E\}R4}
        \begin{itemize}
            \item 4 $\times$ 25Gb/S
            \item Światłowód jednomodowy
            \item Max. odleglość: 10km (L), 40km (E)
        \end{itemize}
    \end{center}
\end{minipage}
\end{tabular}
\end{center}
\pagebreak

\section{IEEE 802.11 Wi-Fi}
    \subsection{Tryby pracy Wi-Fi}
    \begin{center}
    \begin{tabular}{c@{\hspace{2cm}}c}
    \begin{minipage}[t]{0.35\textwidth}
        \begin{center}
            \textbf{Ad-Hoc}
            \begin{itemize}
                \item IBSS (Independent Basic Service Set)
                \item Urządzenia komunikują się ze sobą bezpośrednio
                \item Obecnie już raczej niewykorzystywany
            \end{itemize}
        \end{center}
    \end{minipage}
    &
    \begin{minipage}[t]{0.5\textwidth}
        \begin{center}
            \textbf{Infrastructure}
            \begin{itemize}
                \item BSS (Basic Service Set)
                \item Sieć jest tworzona i koordynowana przez Access Point
                \item Sieć jest identyfikowana przez BSSID (identyfikator AP)
                \item Ruch odbywa się wyłącznie między \textbf{STA}cją a AP (ruch \textbf{STA}$\leftrightarrow$\textbf{STA} realizowany jest w sposób \textbf{STA}$\leftrightarrow$\textbf{AP}$\leftrightarrow$\textbf{STA})
            \end{itemize}
        \end{center}
    \end{minipage}
    \end{tabular}
    \end{center}
    \subsection{Distribution Center (DS)}
        Jest to system łączący Access Pointy ze sobą. Umożliwia on komunikację między stacjami należącymi do dwóch różnych BSS.
        Acces Point składa się z dwóch części:
        \begin{itemize}
            \item \textbf{STA} - z racji przynależności do \textbf{BSS},
            \item \textbf{DS-related} - umożliwia działanie w obrębie \textbf{DS}
        \end{itemize}
    \subsection{Portal}
        Umożliwia komunikację między DS a siecią zewnątrzną pracującą w innym standardzie. Taką siecią może być na przykład sieć Ethernetowa, do której podłączamy Access Pointa.
    \subsection{Extended Service Set (ESS)}
    Na chłopski rozum jest to "sieć WiFi". Bardziej formalnie - zbiór Access Pointów i stacji do nich podłączonych.
    \subsubsection{Sposób identyfikacji sieci}
    \begin{itemize}
        \item \textbf{BSSID} - identyfikator konkretnego Access Pointa należącego do ESS
        \item \textbf{ESSID} - "nazwa sieci", do 32 bajtów UTF-8
    \end{itemize}
    Sposób określenia \textbf{BSSID} na podstawie \textbf{ESSID} nie jest określony przez standard - najczęściej decyduje moc sygnału.
    \begin{figure}
        \begin{center}
            \includegraphics[scale=2]{assets/ESS.png}
        \end{center}
        \caption{Wizualizacja sieci WiFi}
    \end{figure}
    \subsection{Sposoby wyszukiwania sieci Wi-Fi}
    Stacja kliencka przeszukuje obsługiwane kanały częstotliwościowe
    \begin{itemize}
        \item \textbf{Przeszukiwanie pasywne} - stacja nasłuchuje ramek typu \textbf{Beacon}
        \item \textbf{Przeszukiwanie aktywne} - stacja wysyła ramkę typu \textbf{Probe Request} i nasłuchuje odpowiedzi
    \end{itemize}
    \subsection{Ramka w IEEE 802.11}
    \begin{center}
        \includegraphics[width=0.8\textwidth]{assets/wifi-frame.png}
    \end{center}
    Wyróżniamy 3 typy ramek:
    \begin{itemize}
        \item Ramki danych
        \item Ramki kontrolne: RTS, CTS, ACK, Block ACK
        \item Ramki zarządzające: Beacon, Probe Request/Response, Auth/Deauth
    \end{itemize}
    \subsubsection{Po co nam aż 4 adresy???}
    Ramki możemy sklasyfikować na podstawie odbiorcy i nadawcy na:
    \begin{itemize}
        \item STA $\rightarrow$ STA - używane wyłącznie w trybie Ad-Hoc
        \item AP $\rightarrow$ STA
        \item STA $\rightarrow$ AP
        \item AP $\rightarrow$ AP
    \end{itemize}
    A przecież w praktyce możemy wysyłać dane ze STAcji do STAcji podłączonej do innego Access Pointa, w obrębie tego samego ESS.
    Stąd wyróżniamy adresy:
    \begin{itemize}
        \item urządzenia które utworzyło ramkę (SA, Source Address)
        \item adresata utworzonej ramki (DA, Destination Address)
        \item nadającego ramkę (TA, Transmitter Address)
        \item odbierającego ramkę (RA, Receiver Address)
    \end{itemize}
    \begin{center}
    \begin{tabular}{|c|c|c|c|c|c|}
    \hline
    \textbf{To DS} & \textbf{From DS} & \textbf{Address 1} & \textbf{Address 2} & \textbf{Address 3} & \textbf{Address 4} \\
    \hline
    0 & 0 & RA = DA & TA = SA & IBSSID & - \\
    0 & 1 & RA = DA & TA = BSSID & SA & - \\
    1 & 0 & RA = BSSID & TA = SA & DA & - \\
    1 & 1 & RA & TA & DA & SA \\
    \hline
    \end{tabular}
    \end{center}
    \begin{center}
        \textit{Tu moze wstawie jakas wizualizacje tego xd}
    \end{center}
    \pagebreak
    \subsection{Dostęp do medium}
    \subsubsection{DCF}
        Jest to rywalizacyjny algorytm (CSMA/CA)
        \begin{enumerate}
            \item Jeśli medium jest wolne, to po prostu nadajemy co chcemy nadać.
            \item Jeśli zajęte, to generujemy losowe opóźnienie
            \item Opóźnienie to następnie odliczamy tylko, gdy medium jest wolne przez co najmniej czas DIFS
            \item Po odczekaniu rozpoczynamy transmisję
        \end{enumerate}
        \begin{center}
            \includegraphics[width=0.6\textwidth]{assets/dcf.png}
        \end{center}
    \subsubsection{PCF}
        Algorytm kontrolowany. Przydatny podczas dużego obciązenia sieci - np. jedna stacja pobiera Shreka. Wtedy AP przełącza się w tryb PCF na określony czas wysyłając ramkę Beacon.
        \begin{enumerate}
            \item Przełączenie w tryb PCF - Ramka Beacon aka. "teraz mówicie tylko jak was spytam, czy chcecie coś powiedzieć"
            \item AP pyta kolejne stacje, czy chcą coś powiedzieć
            \item Jeśli stacja ma coś do nadania, to nadaje, a AP potwierdza odbiór ramką ACK.
            \item Powrót do trybu DCF.
        \end{enumerate}
        \begin{center}
            \includegraphics[width=0.8\textwidth]{assets/pcf.png}
        \end{center}
    \pagebreak
    \subsection{Zjawisko stacji ukrytych}
    Sytuacja, w której stacje które wzajemnie siebie nie słyszą zakłócają się nawzajem.
    \begin{figure}
        \begin{center}
            \includegraphics[width=0.5\textwidth]{assets/hidden-stations.png}
            \caption{\textbf{A} nie słyszy \textbf{C}, ale zakłóca jego wiadomość skierowaną do \textbf{D}.}
        \end{center}
    \subsubsection{Rozwiązanie}
    Ramki \textbf{RTS} (Request to Send) i \textbf{CTS} (Clear to Send).
    Stacja nadająca najpierw nadaje ramkę \textbf{RTS}, a odbiorca odpowiada ramką \textbf{CTS}. W powyższym przykładzie stacja \textbf{A} nie będzie nadawać, bo po odebraniu ramki \textbf{CTS} wie, że ktoś w pobliżu nadaje, mimo że samej transmisji nie słyszy.
    \end{figure}
\pagebreak
\section{Łączenie sieci}
    Podstawowe problemy:
    \begin{itemize}
        \item Różne technologie (np. Ethernet, Wi-Fi, ATM, Token Ring)
            \begin{itemize}
                \item Do łączenia węzłów sieci WAN ze sobą wykorzystujemy praktycznie tylko Ethernet, więc problem odpada
            \end{itemize}
        \item Różne rozmiary ramek (Większą ramkę trzeba podzielić na mniejsze, by dopasować do innej technologii)
            \begin{itemize}
                \item Ramka w Wi-Fi może zawierać 2342 B danych, a Ramka Ethernetowa tylko 1500 B.
            \end{itemize}
        \item Różna Endianowość
        \item Różne sposoby liczenia sum kontrolnych (trzeba przeliczać na nowo)
        \item Priorytety (Niektóre standardy je mają, inne nie)
    \end{itemize}
    Potrzebujemy jakiegoś magicznego urządzenia, które rozwiąże te wszystkie problemy.
    \subsubsection{Problemy przy przyłączaniu do sieci WAN}
    \begin{itemize}
        \item Trzeba zadbać o to, by ruch rozgłoszeniowy z sieci LAN nie wychodził do internetu
        \item Przepustowość sieci WAN jest ograniczeniem. Sieć w domu ma gigabit na 5 urządzeń, a sieć WAN ma 400Gb na miliardy urządzeń. Przepustowość przypadająca na pojedyńczy węzeł jest znacznie mniejsza.
        \item Routery w sieciach LAN ciężko nazwać faktycznie routerami. Ich zadanie ogranicza się do filtrowania ruchu rozgłoszeniowego. Każdy ruch na zewnątrz wypuszczają w jedno miejsce - na router dostawcy internetowego.
    \end{itemize}
    \subsection{Urządzenia do łączenia sieci}
    \subsubsection{Regenerator}
    Proste, analogowe urządzenie pierwszej warstwy. Jego zadanie ogranicza się do naprawy stłumionego sygnału. Wprowadza znikome opóźnienie, a znacznie wydłuża zasięg sieci.
    \subsubsection{Koncentrator (HUB)}
    Jest to wieloportowy regenerator. Sygnał, który dostaje wysyła wzmocniony na wszystkie porty z wyjątkiem nadawcy.
    \begin{itemize}
        \item Wprowadza znikome opóźnienie,
        \item Jest podatne na kolizje
    \end{itemize}
    To czyni go świenym urządzeniem dla sieci o bardzo małym ruchu (rzędu 3-5\% przepustowości), na przykład w automatyce.
    \subsubsection{Most}
    Urządzenie warstwy pierwszej i drugiej. Posiadają bufor, dzięki czemu są "odporne" na kolizje.

    Most przeźroczysty prosty działa jak koncentrator, z tą różnicą, że w przypadku kolizji ramki są buforowane.

    Most uczący się natomiast wie, pod którym portem jest dane urządzenie i dzieli domenę kolizyjną.

    \paragraph{Problem: sztormy, multiplikacje, pętle ramek rozgłoszeniowych}

    \paragraph{Rozwiązanie: Algorytm Spanning Tree (STP) (IEEE 802.1d)}
    \begin{enumerate}
        \item Wybieramy korzeń drzewa - najczęściej urządzenie z najniższym adresem MAC
        \begin{itemize}
            \item Każdy nadaje informacje, że jest rootem
            \item I każdy przyjmuje do świadomości informacje, że ten o najniższym adresie jest rootem
        \end{itemize}
        \item Każdy z węzłów blokuje porty nie będące drogą do roota
    \end{enumerate}
    A dlaczego nie możemy zbudować drzewa już na poziomie kabli? Możemy, ale dzięki nadmiarowym łączom zwiększamy niezawodność sieci - w przypadku awarii któregoś z łączy budujemy drzewo od nowa. Rapid STP (wprowadzony w standardzie 802.3w) pozwala na szybszą rekonstrukcję drzewa, poprzez zapamiętywanie kilku najniższych MACów.
    \subsubsection{Przełączniki}
    Możnaby je nazwać wielomostami. Likwidują kolizję poprzez rozdzielenie domeny kolizyjnej. W przeciwieństwie do mostów działają równolegle.
    \begin{figure}
        \begin{center}
            \includegraphics[width=0.4\textwidth]{assets/bridge-vs-switch.png}
        \end{center}
        \caption{Porównanie działania mostu i przełącznika}
    \end{figure}
    \paragraph{Architektura przełącznika}
    \begin{enumerate}
        \item Pamięć na tablicę adresów (adres $\rightarrow$ port)
        \item Bufor na ramki
    \end{enumerate}
    \subparagraph{Techniki buforowania}
    \begin{itemize}
        \item Buforowanie wejściowe - każdy port wejściowy ma swój bufor - ryzyko zatoru
        \item Buforowanie wyjściowe - każdy port wyjściowy ma swój bufor - ryzyko wywłaszczenia przez pojedyńczego nadawce
        \item Buforowanie ścieżki - osobne bufory po obydwu stronach. Drogie, nieefektywne, wprowadzające duże opóźnienie.
    \end{itemize}
    \subparagraph{Architektura krzyżowa} - 
    Najstarsza z używanych architektur. Tania w implementacji, aczkolwiek blokująca się. Wymaga stosowaniu osobnych pamięci dla każdego z portu.
    \begin{center}
        \includegraphics[width=0.5\textwidth]{assets/cross-architecture.png}
    \end{center}
    \subparagraph{Architektura ze współdzieloną pamięcią}
    \begin{itemize}
        \item Jedna duża pamięć przechowująca wszystkie ramki,
        \item Dostęp do pamięci może być równoległy albo szeregowy
        \item Dla optymalnej wydajności stosuje się bardzo długą szynę - np. rzędu 1.5kB, tak by móc pobrać całą ramkę w jednym cyklu.
    \end{itemize}

    \subparagraph{Tryby pracy - jak traktować ramki broadcastowe?}
    \begin{itemize}
        \item \textbf{Express Switching} - zarządca decyduje, gdzie wysyłać ruch rozgłoszeniowy. Wymaga konfiguracji ze strony administratora.
        \item \textbf{Transparent Bridging} - ruch rozgłoszeniowy leci na wszystkie porty, niekoniecznie potrzebnie zalewając sieć
    \end{itemize}
    \subparagraph{Metody przełączania}
    \begin{itemize}
        \item \textbf{Cut-Through Fast-Forward} ramka jest przesyłana do odbiorcy odrazu po odczytaniu adresu
        \item \textbf{Store-and-Forward} ramka przed dalszym wysłaniem jest sprawdzana pod kątem poprawności
    \end{itemize}
    \subsubsection{Router}
    Przełącznik dzielący domenę rozgłoszeniową.
    \subsection{Kryteria wyboru przełączników}
    \begin{itemize}
        \item Wspierane standardy: IEEE 802.x, propietary standardy Cisco, \dots
        \item Elastyczność - możliwość konfiguracji
        \begin{itemize}
            \item Autonegocjacja, jakie technologie wspiera?
            \item Port mirroring - do celów diagnostycznych
        \end{itemize}
        \item Przepustowość całkowita. Jeśli mniejsza niż $2\times\mbox{liczba portów} \times \mbox{przepustowość portu}$ to switch będzie bottleneckiem.
        \item Obecność i rozmiar tablicy adresów
        \item Niezawodne zarządzanie - np. dedykowany port do \textbf{SNMP} do wykorzystania podczas gdy sieć jest zalana.
        \item Wsparcie dla VLANów (o których w dalszej części opracowania)
        \item Szybkość przetwarzania ramek - miara $pps$ nie określa rozmiaru pakietów.
        \item Zasięg - czy jest możliwość wpięcia światłowodu?
        \item Kontrolowanie przepływu ruchu
        \begin{itemize}
            \item Dostępność Express Switching
            \item Możliwość limitacji przepustowości dla ruchu broadcast
            \item Sztuczne blokowanie medium w sytuacjach krytycznych (gdy np. wiele kilientów łaczy się do jednego serwera)
            \item MAC Address whitelisting
        \end{itemize}
        \item Wsparcie dla agregacji łączy
    \end{itemize}
    \subsection{Agregacja łączy w standardzie \textbf{IEEE 802.3ad}}
    Wykorzystanie kilku łączy full-duplex o tej samej prędkości do realizacji jednego połączenia. Zwiększa wydajność, a także podnosi niezawodność (gdy jedno łącze ulegnie awarii, to pozostałe dalej potrzymują łączność)
    \begin{center}
        \includegraphics[width=0.5\textwidth]{assets/aggregation.png}
    \end{center}
    Z powodu problemu z kolejkowaniem ramek, nie jest możliwe aby ruch między dwoma węzłami przechodził przez kilka łączy na raz. Z tego też powodu nie zwiększy to przepustowości między dwoma konkretnymi hostami, zwiększy natomiast ogólną przepustowość sieci.
\pagebreak

\section{Wirtualne sieci lokalne \textbf{VLAN} - \textbf{IEEE 802.1Q}}
    \textbf{VLAN}y umożliwiają tworzenie kilku sieci lokalnych z wykorzystaniem tego samego sprzętu.

    \paragraph{Idea: VLAN = Domena rozgłoszeniowa}

    Ogranicza to role routerów wyłacznie do realizacji połączeń między sieciami. Nie muszą już odfiltrowywać ruchu rozgłoszeniowego.

    \subsection{Po co separować sieć?}
    \begin{itemize}
        \item bezpieczeństwo - niemożliwość podsłuchiwania ruchu poza VLANem
        \item odseparowanie bardziej obciążonych sieci od reszty
        \item Rozgłoszenia w mniejszym stopniu zalewają sieć
    \end{itemize}
    \subsection{Określenie przynależności do VLANa}
    \begin{itemize}
        \item Grupowanie portów (segment wirtualny)
        \begin{itemize}
            \item Działa w warstwie pierwszej, nie wprowadza opóźnienia
            \item Uniemożliwia przypisanie jednej stacji do kilku VLANów
        \end{itemize}
        \item grupowanie adresów (podsieć wirtualna)
        \begin{itemize}
            \item MAC (warstwa druga)
            \item IP (warstwa trzecia)
        \end{itemize}
        \item grupy multicastowe
        \item reguły logiczne
    \end{itemize}
    \subsection{Przesył informacji o przynależności do VLANa między przełącznikami}
    W tej kwestii przełączniki różnych firm są często niekompatybilne ze sobą.
    \begin{itemize}
        \item Przesyłanie tabel (informacji o VLANach) między przełącznikami. Problematyczne - tabele mogą być duże, jak często trzeba je aktualizować?
        \item Oznaczanie ramek - \textbf{IEEE 802.1Q}
        \begin{itemize}
            \item Dodatkowe 4-bajtowe pole w ramce Ethernetowej (Zmniejsza minimalny rozmiar danych do 42 bajtów)
            \item W tym polu 12-bitów na numer VLANa i 3 na ustalenie priorytetu (QoS)
            \item Maksymalny rozmiar ramki wzrasta do 1518 bajtów - starsze urządzenia mogą taką odrzucać z automatu
            \item Maksymalnie 4094 VLANy - w specyficznych przypadkach zbyt mało - np. w kompleksie hotelowym, gdzie chcemy odseparować każdy z pokoi
        \end{itemize}
    \end{itemize}
    \subsection{Klasyfikacja VLANów}
    \begin{center}
    \begin{tabular}{c@{\hspace{2cm}}c}
    \begin{minipage}[t]{0.4\textwidth}
        \begin{center}
        Liczba przełączników
        \end{center}
        \begin{itemize}
            \item wewnętrzny - w obrębie jednego przełącznika
            \item zewnętrzny - w obrębie kilku przełączników
        \end{itemize}
    \end{minipage}
    &
    \begin{minipage}[t]{0.4\textwidth}
        \begin{center}
        Świadomość użytkownika
        \end{center}
        \begin{itemize}
            \item ukryty - użytkownik nie wie, że jest w VLANie
            \item jawny
        \end{itemize}
    \end{minipage}
    \end{tabular}
    \end{center}
\pagebreak
\section{IPv4}
    \subsection{Cechy IP jako protokołu warstwy 3}
    \begin{itemize}
        \item Niezależne od technologii warstw niższych (internet działa tak samo, niezależnie czy używamy Wi-Fi czy Ethernetu)
        \item Zasięg światowy
        \item Potwierdzenia End-to-End
        \item Zaimplementowane typowe aplikacje - Poczta, Transfer plików
    \end{itemize}
    \subsection{Nagłówek IPv4}
    \begin{figure}
        \begin{center}
            \includegraphics[width=0.7\textwidth]{assets/ipv4-header.png}
        \end{center}
        \caption{Struktura nagłówka IPv4}
    \end{figure}
    \begin{itemize}
        \item Wersja protokołu - aż 4 bity, a funkcjonują tylko dwie wersje
        \item ID datagramu - wraz z adresem nadawcy w założeniu tworzy indywidualny w skali internetu identyfikator. W praktyce pole to może być za małe - w tranmisjach multimedialnych.
        \item TTL - dekrementowany co hop o nieokreśloną wartość
        \item Suma kontrolna samego nagłówka
        \begin{itemize}
            \item Przy obecnym BER szansa uszkodzenia nagłówka znikoma
            \item Co hop musi być rekalkulowana (duże opóźnienia)
            \item Obecnie zbędna, bo wszechobecny Ethernet i tak dodatkowo liczy sumy kontrolne
        \end{itemize}
        \item Adresy - są za krótkie, by jednoznacznie zidentyifkować wszystkie hosty w internecie
        \item Istnieje opcjonalne, 4 bajtowe pole które nie jest nigdzie wykorzystywane
        \item Fragmentacja
        \begin{itemize}
            \item Pole znaczników
            \begin{itemize}
                \item "Zarezerwowany" bit - zawsze ustawiony na 0
                \item \textbf{DF} (Don't Fragment) - zakaz fragmentacji
                \item \textbf{MF} (More Fragments) - ustawiany na 0 tylko dla ostatniego fragmentu
            \end{itemize}
            \item Zwiększa zawodności sieci
            \item Zwiększa ilość wysyłanych informacji, bo nagłówki trzeba wysyłać wielokrotnie
            \item Zwiększa koszt operacji routingu
            \item Służy do łączenia sieci o różnych MTU (np. przesłanie 2kB ramki z Wi-Fi do Ethernetu)
        \end{itemize}
    \end{itemize}
\pagebreak
\section{Aplikacje sieci TCP/IP}
    \subsection{Domain Name Service (DNS)}
    Umożliwia wykorzystywanie nazw domenowych (łatwych do zapamiętania dla człowieka) zamiast surowych adresów IP.

    \subsubsection{Domeny}
    Jest to drzewiasta struktura, gdzie korzeniem zarządza \textbf{ICANN}. To ta instytucja rejestruje domeny najwyższego poziomu (z ang. Top-Level Domain (\textbf{TLD}))

    \paragraph{Domeny najwyższego poziomu}
    \begin{itemize}
        \item gTLD - general np. .com, .net ($\ge$3 znaki) - zarządzane najczęściej przez jakieś firmy
        \item ccTLD - krajowe np. .pl, .de, .tv (2 znaki) - zarządzane przez instytucje rządowe - w Polsce NASK
    \end{itemize}
    Każda domena musi posiadać co najmniej 2 serwery DNS, po to by w razie awarii jednego z nich drugi mógł przejąć zadanie.
    \subsubsection{Główne serwery DNS \{a-m\}.root-servers.org}
    Ograniczono liczbę głównych serwerów do 13 - bo informacje o większej ilości serwerów nie zmieściłyby się w jednym pakiecie UDP.

    W praktyce na jeden adres IP przypada kilkaset serwerów DNS - wykorzystuje się anycast.

    \subsubsection{Sposoby kierowania zapytań do DNS}
    \begin{itemize}
        \item Rekurencja - Serwer DNS pyta inne serwery w imieniu klienta
        \item Iteracja - klient bezpośrednio odpytuje serwery DNS
    \end{itemize}
    \begin{center}
        \includegraphics[width=0.4\textwidth]{assets/dns-iteration-recurse.png}
    \end{center}
    \subsubsection{Odpowiedzi od DNS}
    \begin{itemize}
        \item Autorytatywne - pochodzące od serwera zarządzającego domeną, o którą jest pytany. Stąd, zawsze jest aktualna
        \item Nieautorytatywne - pochodzące z zewnątrz, buforowane na serwerze przez jakiś czas
    \end{itemize}

    \subsubsection{Rekordy DNS}
    \begin{itemize}
        \item \textbf{SOA} (Start of Authority) - podstawowe informacje nt. domeny
        \item \textbf{NS} (Name Server) - serwer DNS obsługujący domenę
        \item \textbf{A} (Address) - po prostu adres IPv4 (lub \textbf{AAAA} dla IPv6)
        \item \textbf{CNAME} (Cannonical Name) - wskazanie na inną nazwę
        \item \textbf{MX} (Mail Exchange) - Serwer pocztowy obsługujący domenę
        \item \textbf{HINFO} - Informacje o sprzęcie obsługującym domenę
    \end{itemize}

    \subsection{Poczta E-Mail (protokoły SMTP i IMAP/POP)}
    \begin{figure}
        \begin{center}
            \includegraphics[width=0.8\textwidth]{assets/mail.png}
        \end{center}
    \end{figure}
    \section{W jaki sposób komputery rozmawiają ze sobą?}
    Załóżmy, że chcemy wysłać wiadomość o treści "kielnia" do innego komputera.
    Na pozór proste - kilka linijek w Pythonie. Ale co się dzieje pod spodem?
    \subsection{Warstwa aplikacji}
    Mamy nasz komputer, któremu każemy wysłać wiadomość do drugiego komputera.
    \subsection{Warstwa transportowa}
    Komputer w pierwszej kolejności tworzy opakowuje dane do pakietu warstwy transportowej. 
\end{document}